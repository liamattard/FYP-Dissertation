\subsection{Retrieving travel products}

We implemented the Google Maps API as the data source for our application
because of its real-time accuracy and massive dataset compared with the other
approaches and other APIs that we discussed in the literature review
%~\cite{googleSite, iltifat2014generation}. In addition, the nearby search
endpoint allows the app to search
for places of a given category within a specified area. In order to retrieve
the places for the application, eight requests are made, each requesting places
of different categories. To solve the issues with time windows, we split the
endpoints into two categories. Five of the requests represent places shown as
part of the itinerary during the day, and the rest represent places shown
during the night. 

%%TODO: Add figure
%Table X shows the eight categories that were requested. These
%categories are based on the ones used by Wörndl et al.~\cite{Worndl2017} for their
%application.


%In return, the API returns a list of places of the specified area and category
%and attributes about each place. The attributes used by our application include
%the place's name, rating, the number of reviews and the coordinates. All of
%these attributes help the application further optimise the algorithm to find
%the perfect itinerary. 

%%TODO: Add figure
%Figure X shows an example of a response from the API.\

