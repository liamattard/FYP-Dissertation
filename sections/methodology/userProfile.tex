
\subsection{Generating the User Profile}
Social media's effect on the world is something significant~\cite{Miller2016}.
That is why this application builds a user profile from
the user's social media. 

The application built by Lim et al.~\cite{Lim2018a} allowed the user to connect
the application with their Flickr profile to scan their past trips. However,
Facebook provides an API that would allow users to connect both their Facebook
and Instagram accounts and request content from the user with their permission.
A significant advantage is that the API allows the
application not to limit the results to mimic only
past user's trips like the application by Lim et al.~\cite{Lim2018a} and gather
preferences from his complete profile.
The app requests two things from the
potential tourist's social media, the photos and the liked pages and tries to
classify these into six categories that make up the user's travel interest
vector; 

\begin{center}
    [   
    1 Beach,
    2 Nature,
    3 Shopping,
    4 Museums,
    5 Clubbing,
    6 Bars ]
    
\end{center}



These categories are the same categories that we requested from the google maps
API except `cafeterias' and `restaurants'. These two categories were not
included because the application tries to suggest the best places to eat as
part of the timetable, irrelevant to the user's profile. At the start of the
application, the app initialises all vector values to zero and increments a
value whenever the user's content matches a category. We will describe how the
app classifies both the user's liked pages and the user's photos separately in
the upcoming subsections.

\subsubsection{Transforming the liked pages into the travel interest vector}

The Facebook API allows the application to request each category of the user's
liked Facebook pages. The API's documentation contains a whole list of possible
categories. 
%TODO: Add Url
%https://www.facebook.com/pages/category. 

The app iterates through
all of these user's likes categories and increments a value in the user's
vector whenever the Facebook result matches one of the six travel interest
vector values. For example, if a user likes a page with class `DJ', the user's
clubbing vector value is incremented, and if a page is labelled as a
`Mountain', the app increments the user's nature vector value.

\subsubsection{Transforming the user's photos into the travel interest vector}


Convolutional Neural Networks are used to classify an
image because of their high accuracy.\@
%TODO: Cite Zhao

Therefore, we decided to test out two approaches for
classifying the photos into the app's six categories. 

Since the places365 model is not specifically trained
on the six categories that this application is based
on, we wanted to compare a model directly on trained
on the categories with places 365 models. 	

