
\subsection{Generating the User Profile}
Social media's effect on the world is something significant~\cite{Miller2016}.
That is why this application builds a user profile from
the user's social media. 

The application built by Lim et al.~\cite{Lim2018a} allowed the user to connect
the application with their Flickr profile to scan their past trips. However,
Facebook provides an API that would allow users to connect both their Facebook
and Instagram accounts and request content from the user with their permission.
A significant advantage is that the API allows the
application not to limit the results to mimic only
past user's trips like the application by Lim et al.~\cite{Lim2018a} and gather
preferences from his complete profile.
The app requests two things from the
potential tourist's social media, the photos and the liked pages and tries to
classify these into six categories that make up the user's travel interest
vector; 

\begin{center}
    [   
    1 Beach,
    2 Nature,
    3 Shopping,
    4 Museums,
    5 Clubbing,
    6 Bars ]
    
\end{center}



These categories are the same categories that we requested from the google maps
API except `cafeterias' and `restaurants'. These two categories were not
included because the application tries to suggest the best places to eat as
part of the timetable, irrelevant to the user's profile. At the start of the
application, the app initialises all vector values to zero and increments a
value whenever the user's content matches a category. We will describe how the
app classifies both the user's liked pages and the user's photos separately in
the upcoming subsections.

\subsubsection{Transforming the liked pages into the travel interest vector}

The Facebook API allows the application to request each category of the user's
liked Facebook pages. The API's documentation contains a whole list of possible
categories. 
%TODO: Add Url
%https://www.facebook.com/pages/category. 

The app iterates through
all of these user's likes categories and increments a value in the user's
vector whenever the Facebook result matches one of the six travel interest
vector values. For example, if a user likes a page with class `DJ', the user's
clubbing vector value is incremented, and if a page is labelled as a
`Mountain', the app increments the user's nature vector value.

\subsubsection{Transforming the user's photos into the travel interest vector}


Convolutional Neural Networks have become a standard
for classifying an image because of their high
accuracy~\cite{Zhou2018}. Therefore, we decided to test
out two approaches for classifying the photos into the
app's six categories. 

Zhou et al.~\cite{Zhou2018}.trained several CNNs on the
places365-standard  dataset of about 1.8 million
images to classify an image into 365 different scene
categories. However, the places365 model is not
specifically trained on the six categories of our
application. Therefore, we need to carefully map the
365 categories with our six application's categories.
We also introduced a Tensorflow Keras
sequential model, explicitly trained on the six
application's categories to compare.

We used the Resnet places365 models, Resnet-18 and
Resnet-50 since they achieved the highest top-5
validation accuracy on the places365 dataset. The
Resnet 18 comprises 18, and the Resnet 50 comprises 50
convolutional layers. They both converge an output
layer representing the 365 output categories.  Figure
\ref{Resnet} shows a summary of the whole Resnet 18 model. 


The Tensorflow Keras Sequential model comprises three
convolutional layers with a rectified linear unit
(ReLu) activation function. A pooling layer follows
each to lower the input volume's spatial dimension for
the upcoming layers. The final layer represents a
flattening layer and two dense layers to reduce the
outputs to the six application categories, and another
layer representing the `None' classification. Figure~\ref{Keras}
shows a summary of the whole model. The dataset
comprises X public internet images representing the
seven classes: Beach, Nature, Museums, Shopping,
Clubbing, Bars and None.

\begin{figure}[!tbp]
  \centering
  \begin{minipage}[b]{0.4\textwidth}
    \includegraphics[width=0.8\textwidth]{KerasSequential.png}
    \caption{Keras Sequential Architecture Summary}
    \label{Keras}
  \end{minipage}
  \hfill
  \begin{minipage}[b]{0.4\textwidth}
\includegraphics[width=0.8\textwidth]{Resnet.png}
\caption{Resnet 18 Architecture Summary}
\label{Resnet}
  \end{minipage}
\end{figure}

