
\subsection{Producing the activity plan}

After the app generates the dataset of POIs and the
user's travel interest vector, we formulate an
efficient activity plan using these two inputs. This
itinerary generator is based on the existing state of
the art activity planners~\cite{Sylejmani2017, Wisittipanich2020}
with some adjustments: We wanted the trip's output to take the
form of an itinerary.  The problem takes the form of
the `day' and `night'  category split discussed in
the literature review.  The scoring of itineraries is adjusted
with the travel interest vector.

The problem definition of our novel itinerary planner
is mathematically formulated as follows. A tourist trip is made up
of some pre-defined user constants alongside the travel interest
vector. The predefined constants are:
\\


\setlength{\tabcolsep}{20pt}

\begin{tabular}{l l}

\textit{M}:  &  The number of travelling days \\
\textit{C}: & The activity pace (ie. the greater the \\
 & C value, the more activities are generated in \\
 & a day) \\  
\end{tabular}
\\
\\
\\
The objective function of our itinerary planner is:

\[ \text{MAX}  \sum_{m=0}^{M} ( S_{{D_m}} + S_{{E_m}}) \]
where:
\\
\begin{tabular}{l l}
% \textit{i,j}:  &  POI (\textit{i,j} = 2,3,...,\textit{N}) \\
\textit{m} & Travelling day (\textit{m}=1,2,\ldots, textit{M}) \\ 
\textit{$D_m$} & Morning section of day number m \\  
\textit{$E_m$} & Evening section of day number m \\  
\textit{$S_{D_m}$} & Score of the morning section $D_m$ \\  
\textit{$S_{E_m}$} & Score of the evening section $E_m$ \\  
\end{tabular}
\\
\\

A day is made up of the morning ${D_m}$ section and
the evening ${E_m}$ section. The timetable suggests a
POI in the morning, then somewhere to eat, and the
rest is dependant on the activity pace C. That is why
the morning section is made up of $C + 2$ tourist
attractions. The evening section suggests a place to
eat and a POI; therefore, the evening section is just
made up of $2$. 


\[D_m = Y_i + Y_f + C ( Y_i) \]
\[E_m = Y_f + Y_j \]

\begin{tabular}{l l}

\textit{i} & Morning Tourist attraction (i = 1, 2, 3,…, $n_1$)\\
\textit{j} & Evening Tourist attraction (j = 1,2, 3,…, $n_2$)\\
\textit{f} & Food Place (f = 1,2, 3,…, $n_3$)\\
\textit{$Y_{i|f|j}$}: & 1 if a tourist visit attraction i, j or f and  0 if otherwise\\
\end{tabular}
\\ 
\\
\begin{tabular}{l l}
\textbf{Constraints} & \\
\textit{$ \sum_{m=0}^{M}\sum_{i=0}^{n_1}{Y_i} \leq 1$} & Ensures that all morning tourist attractions are \\ & not visited more than once throughout the whole \\ & itinerary\\

\textit{$ \sum_{m=0}^{M}\sum_{j=0}^{n_1}{Y_j} \leq 1$} & Ensures that all evening tourist attractions are \\ & not visited only once throughout the whole \\ & itinerary\\


\end{tabular}

\subsubsection{Calculation of Score}

The score $S_{D_m}$ or $S_{E_m}$ is calculated using

\[ S_{D_m | E_m} = \frac{1}{T} + R + V\]

where:
\\
\begin{tabular}{l l}
% \textit{i,j}:  &  POI (\textit{i,j} = 2,3,...,\textit{N}) \\
\textit{T} & Total distance between each tourist attractions in the \\ & morning/evening of day m\\ 
\textit{R} & Average rating of the tourist attractions in the \\ & morning/evening of day m \\  
\textit{V} & how much the tourist attractions of the \\ & morning/evening of day m match with the user's \\ & travel interest vector \\  

\end{tabular}
\\
\\


\subsection{Retrieving travel products}

We implemented the Google Maps API as the data source for our application
because of its real-time accuracy and massive dataset compared with the other
approaches and other APIs that we discussed in the literature review
%~\cite{googleSite, iltifat2014generation}. In addition, the nearby search
endpoint allows the app to search
for places of a given category within a specified area. In order to retrieve
the places for the application, eight requests are made, each requesting places
of different categories. To solve the issues with time windows, we split the
endpoints into two categories. Five of the requests represent places shown as
part of the itinerary during the day, and the rest represent places shown
during the night. 

%%TODO: Add figure
%Table X shows the eight categories that were requested. These
%categories are based on the ones used by Wörndl et al.~\cite{Worndl2017} for their
%application.


%In return, the API returns a list of places of the specified area and category
%and attributes about each place. The attributes used by our application include
%the place's name, rating, the number of reviews and the coordinates. All of
%these attributes help the application further optimise the algorithm to find
%the perfect itinerary. 

%%TODO: Add figure
%Figure X shows an example of a response from the API.\



%\subsubsection{}
\subsubsection{Optimisation Algorithms}

The following will discuss the steps required to
produce the two timetable optimisation algorithms. PSO
and GAs are two meta-heuristics that use a population
to converge to a fit solution. Therefore, they require
an initial random generation of possible timetables. In
our algorithm, we introduce a method of randomisation
bias which will be discussed before the algorithms.


\paragraph{Random Bias}
With this technique, the randomness of the initial
population is weighted based on two components.
\begin{enumerate}

    \item  the place's rating and 

    \item the place's number of ratings.
\end{enumerate}


Before the algorithm starts each POI will be given a score 
that will determine its probability of being chosen as
part of the initial population. For example, if a
POI $i$ has a rating, 
$r_i$ which a value from 0 to 5, and
the number of ratings from users $z_i$. The score of $i$
is calculated using the following equation:

    \begin{center}
        score = $\frac{r_i}{5}+\frac{z_i}{\text{max}(\sum_{j=0}^{N}{z_j})}$

    \end{center}
%\[ \text{score} = (p_i)p_i\text{.rating}\]

This bias gives a head start to the algorithm
rather than just starting optimising from purely
random itineraries, highly likely to be of bad
quality.


\begin{algorithm}[h]

 iter = x \;
 count = 0 \;
 particles = RandomBiasInitialiser(); \;
 bestParticle = particle[0] \;

 \While{count $>$ x}{

     i = 0 \;
     maxScore = 0 \;

     \While{i $<=$ particles.length}{

         \tcc{Updating personal best}
        \If{particles[i].score $>$ particles[i].personalBest.score}{

             particles[i].personalBest = particles[i].position\;

         }

         \tcc{Updating global best}
        \If{particles[i].score $>$ maxScore}{


             maxScore = particles[i].score \;
             bestParticle = particles[i].position \;

         }

         i = i +1 \;

         \tcc{Calculate new position}
         particle[i].calculateNewPosition() \;



     }
     \Return bestParticle

 }

 \caption{Particle Swarm Optimisation}
    \label{PSOAlgorithm}
\end{algorithm}



\paragraph{Particle Swarm Optimisation}

In PSO, the whole population is referred to as the
swarm, whilst a single member a particle. Each
particle has a 2-dimensional position(P) vector
representing the current timetable solution and a
2-dimensional velocity(V) vector expressing the direction
of the particle during its search period. 

The algorithm has six integer parameters including the
number of particles and the number of iterations. The
personal acceleration (PA) affects how far away the
particle moves from the personal best position (PB). 
The global best acceleration (GA) attracts the global
best position (GB) of the whole swarm. 
The inertia (I) constant helps the particle
explore new solutions and escape the local minima
through randomness. 

At each
iteration, the new velocity is calculated using
\begin{center}
    new velocity = I + (PA * (PB - P)) + (GA * (GB - P))
\end{center}

the new position is calculated using
\begin{center}
    new position = P + new velocity 
\end{center}

After a few iterations have passed, particles use
their velocity and move towards the optimum position.
We demonstrate the framework of our PSO algorithm in
algorithm \ref{PSOAlgorithm}.


\paragraph{Genetic Algorithms}

Genetics algorithms use biological terms to describe
their attributes. For example, a timetable solution in
population is referred to as a chromosome~\ref{Abbaspour2011}. 

In PSO, the algorithm optimises by allowing each
particle to move closer to the global best every
iteration. In comparison, in GAs, first, the best
chromosomes known as the elites are selected from each
iteration. Then, three techniques, namely selection,
mutation, and crossover, are applied to generate the
next population.

We used the geneticalgorithm2 package~\footnote{https://pypi.org/project/geneticalgorithm2/},
which allowed us to use the same score function and
the random bias to initialise the particles. The
algorithm has 7 parameters. The parents portion
represents the number of parents who will reproduce
and create the next generation. The mutation
probability determines the chance a POI in a
chromosome will be replaced by a random value to which
the algorithm will converge less quickly and explore
more of the search space. The crossover probability
will affect the chance that part of its solution goes
to the child. Finally, the elite ratio determines how
much of the best chromosomes in an iteration make it
to the next iteration.  There are many types of
crossovers techniques. In this algorithm, we
explored \textit{one point, two point, uniform and shuffle crossover}.  The algorithm produced the following
steps: \\
\textbf{Step 1}: Initialise the first population using random bias. \\
\textbf{Step 2}: Select the best chromosomes from the population. \\ 
\textbf{Step 3}: Select the elite particles that will make it to the next iteration.\\
\textbf{Step 3}: Apply Crossover, Mutation and Selection on the population. \\
\textbf{Step 4}: Check if the number of iterations has exceeded. 

