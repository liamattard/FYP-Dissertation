\subsection{Methods of retrieving travel products}

Before producing an itinerary, RSs have to formulate a dataset of POIs from
some data source. The proposed tour recommendation algorithm will then evaluate
a guided path, route or itinerary from this dataset after understanding the
users' implicit preferences such as the travel date and activity moderation. There are several ways to identify an appropriate
data source representing real-life tourist trajectories.  


\paragraph{Geotag mining:} One approach is made by gathering tourist products by mining them from
geotagged images of Location-Based Social Networks (LSBN) such as Flickr,
Facebook or Twitter~\cite{DeChoudhury2010, Memon2015, Lucchese2012, Lim2018a,
HuiLim, HuiLima, Kurashima2013, Kurashima2010, Brilhante2013, Brilhante2015 }.
Lim et al.~\cite{Lim2018} denote this process into three steps; 


\begin{enumerate}

\item First, the application assembles an organised series of relevant
    photographs of the user's destination from the LSBN.\@

\item The application then maps these pictures with a list of popular places
    extracted from sites like Wikipedia.

\item Since the photos contain metadata, like the location and the timestamps;
    the application can calculate an approximate visit duration for each
    specific POI.\

\end{enumerate}

\paragraph{GPS-based data sources:}
The ubiquitous presence of smartphones and
GPS-enabled devices has facilitated
another approach to collecting trajectories~\cite{10.1145/1889681.1889683,
10.1145/1526709.1526816, Chen2011a}. A system
can automatically gather the best POIs to visit based on other users'
historical paths providing additional information such as the average time
people spend at a specific POI and how many people go there. However, privacy
issues are the main caveat towards this approach since it requires people to
share their location constantly and publically\cite{Lim2018}.

\paragraph{Prebuilt dataset:} The most straightforward method is done by self-defining the POIs or gathering
them from a dataset such as the TSPLIB95~\footnote{Sample instances for
travelling salesman Problem:
http://comopt.ifi.uni-heidelberg.de/software/TSPLIB95/}. Manually collecting
travel products provides precision and a better understanding of the itinerary
that the algorithm will generate. However, the algorithm would be
dataset-specific testified and personalised towards what the authors of the
dataset think are the best POIs to visit in a location~\cite{Chou2021a,
Wisittipanich2020, Erbil}.

\paragraph{Maps APIs:} A prompt and accurate strategy towards gathering essential places in the
vicinity is using Mapping \& Location APIs such as Foursquare's, Google's or
TripAdvisor's. Wörndl et al.\cite{Worndl2017} use this approach and build a
dataset of prominent POIs by querying the API with the user's desired location.
In return, they receive a sequence of places and information about each site,
including its category, other user's ratings, opening hours, coordinates and
helpful additional information to use as criteria for the itineraries. However,
the API does not return the average amount of time people spend at a specific
POI.Wörndl et al.\cite{Worndl2017} solve this issue by adding a fixed time
constant for each category with a variable dependant on the POI's score. For
example, suppose a restaurant's time constant is 45 minutes, and the chosen
restaurant has a high score (based on its rating and user's characteristics).
In that case, the time spent at the restaurant will increase by an additional
15 minutes. A significant advantage of using this approach is that the vast
number of POIs that these endpoints return. According to Google's website, the
API contains up to 200 million places with 25 million updates daily which an
application can achieve with a few REST requests\cite{iltifat2014generation, googleSite}.


