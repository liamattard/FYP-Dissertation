
\subsection{In-depth, semi-structured interviews results and discussion}


The following will describe some of the conversations and information gathered
from the interviews. Then, we will place the main points about the rest of the
discussions in the appendix section. 

From all of the interviews, fifteen people preferred the personalised
itineraries and gave valid reasons. However, three out of the people who
selected the non-personalised itinerary stated that they think their social
media presence does not reflect their travel preferences.

\paragraph{Person one}:  Person one enjoys travelling to fashion-related places
and enjoys shopping. They also find nightlife and clubbing significant. Since
they do not like to miss out on any POIs of a site, they chose to generate a
fast-paced activity plan(3) for seven days. In addition, person one is very
active on social media and frequently posts the activities they visit. The
application gathered 100 preferences from this person, and table X shows her
generated travel interest vector.
									
The first timetable shown was the non-personalised itinerary, and they made the
following points:

\begin{itemize}
\item Initially, they thought the timetable was quite personalised since it tried to capture a balance of POIs of a different category and thought this reflected their lifestyle.
\item They were satisfied with the travel time between the places making it very doable.
\item They felt that the timetable included too many cultural places and stated that they would change about 40% of the timeline if they had to follow it strictly and spend more time shopping.
\item They rated the itinerary as 'quite personalised' and were 'very satisfied'.
\end{itemize}

The following are comments from the second itinerary, i.e. the personalised one:
\begin{itemize}
\item The interviewee changed their opinion immediately and figured out this was the personalised solution since the first day was a day dedicated to shopping. 
\item Since the timetable included more natural sights, they felt a more appealing itinerary and would manage to stick with this plan for about 95%.
\item The suggested places were all places that the person has been to recently.
\item The evening places were very appealing since it includes a lot of the best night clubs.
\item They rated the itinerary as 'very personalised' and were 'very satisfied'.
\end{itemize}
When shown their automatically generated preferences, person one felt that the
'beach' category should be higher, and the rest was very accurate. In addition, person two deemed this approach ideal since it gives an in-depth itinerary for a tourist, which is also ideal to discover new places.


\paragraph{Person two}:  When travelling, person two looks for a relaxed
holiday with a mix of beaches and the main sites. Therefore, they chose to
generate a moderate activity plan(2) for five days. Furthermore, person two
thinks that her social media presence represents her travel preferences since
they always post when they travel and posts pictures of events.  The
application gathered ten pictures and liked pages from their profile, and table
X shows her generated preferences.  The first timetable shown was the
non-personalised itinerary, and they made the following points:
\begin{itemize}
\item The holiday is structured very well and reflects their potential travel plan since there is a resting time between the evening activities and a place to eat right after.
\item They were unsure whether the first itinerary is personalised since they usually spend more time at the beach and less time in Museums.
\item There are too many clubs in the evening and would personally go to more bars and relaxed places.
\item The first itinerary received 'not very personalised' but rated 'satisfied' overall. 

\end{itemize}
The following are comments from the second itinerary, i.e. the personalised one:
\begin{itemize}
\item This itinerary is more personalised since there are more beaches and natural places.
\item The evenings are more adapted since there are more places to sit down, drink, and chat than clubbing venues.
\item This itinerary received  'very personalised' and rated 'very satisfied overall. 
\end{itemize}
After the second itinerary, we discussed the travel interest vector with the
interviewee. They would leave their characteristics the same, however, switch
the shopping category with museums. They mentioned that the museums category
could be a bit too high since they like to view science-related content to keep
up to date. Nonetheless, they would want to visit a lot of museums and
history-related places when on holiday. Six other interviewees also mentioned
that their 'museums' characteristic was rated too high. 

To further personalise the itinerary, person two suggested that the algorithm
should also look at the photos posted by the people they follow online since it
would gather much more information. 

\paragraph{Person Three}: When travelling, they tend to look for places with a
lot of nature and cultural sightseeing. In addition, this person is very
interested in music and art, so they also like to travel to art galleries and
museums.  Person three likes to make the most out of their holiday, so they
chose to generate a fast-paced six-day itinerary. However, they don't think
their social media profile represents what they would look for in a holiday. As
a result, the application only gathered ten items, and their preferences are as
follows:

The first result shown to this person was the non-personalised itinerary, and
they made the following conclusions: 

\begin{itemize} 
    \item They would enjoy
    less nightlife and more calm things in the evening.  
    \item They would like
    more natural sights along with the museums such as hikes and walks around
    the place.  
    \item Felt that the timetable was not very personalised.
    \end{itemize} 

From the non-personalised itinerary, person three made the
following conclusions:

\begin{itemize} \item This timetable contains more of a balance of things.
Therefore they felt it is more appropriate.  \item There are more
nature-related places, so it felt more timetable.  \end{itemize}
  
We gathered from all of the interviews that the more content a person posts,
the more accurate the characteristics are. Figure X shows how most of the
collected preferences belong to users that felt very satisfied with the overall
generated personalised itinerary.

\subsubsection{Revisiting the Objectives}

