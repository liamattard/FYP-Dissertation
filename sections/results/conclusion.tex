\section{Conclusion and Future Work}

In this dissertation, we addressed the question `Can a system automatically
recognises a tourist's travel preferences and use this information to generate
a personalised itinerary for a holiday?'. 

We introduced an automatic preference gathering technique that scans a user's
social media profile and analyses the user's photos and liked pages. Our work
presents a comparison between three CNNs: the Resnet 18, Resnet 50 and Keras
Tensorflow models trained to classify an image into one of the following user
characteristics; Beach, Nature, Shopping, Museums, Nature Clubbing and Bars.

As a result, we have built an application to solve the TTDP that considers the
popularity of POIs, the user's automatically generated interests, the user's
activity pace and the distance between places as constraints to recommend a
suitable itinerary.

We optimised the timetables by comparing two meta-heuristics, Particle Swarm
Optimisation and Genetic algorithms and found that the PSO algorithm provided
the best balance between achieving the highest score and providing the solution
in a reasonable amount of time. 

Finally, we asked 20 people to provide us with feedback regarding the
personalisation of the timetables through in-depth semi-structured interviews.
The interviews displayed personalised results and non-personalised results in
random order, and without informing the interviewees which one was which, 15
people prefered the personalised itinerary.

So far, we used the user's photos and the user's liked pages to gather the
preferences automatically. However, we have seen through the interviews that not
everyone feels that what they post online represents their travel preferences.
Therefore, more diverse approaches through social media should be applied and
compared in future work such as:

\begin{itemize}
\item Gather characteristics from social media contacts that potential tourists follow.
\item Use Natural Language Process techniques to scan the user's online posts.
\item Expand the application to use other social media APIs such as Twitter and Google Photos to reach more people. 
\end{itemize}

Although there is always room for further development, this dissertation forms
the basis to prove that an automatic preference gathering system can further
assist a travel-related application to achieve more satisfactory results for
its users.

