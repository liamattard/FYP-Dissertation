\section{Background Research and Literature Review}

This section aims to position our study and ease the
understanding of its conclusions and implications.
Therefore, the upcoming sections describe both of the
technologies that we will use in this dissertation. In
the first part, we give an overview of the TTDP
research area, and in the second part, we aim at
clarifying existing research regarding automatic user
profiling.

\subsection{Recommender Systems}

We introduce the term Recommender Systems (RS) as a
solution for the TTDP and present background knowledge
and relevant work that forms this thesis's basis.
Recommender systems offer tourists information in a
unified and centralised way, providing them with a
plan for their trip~\cite{Santamaria-Granados2020,
DiBitonto2010a, Lim2018}. Two domains develop current
RS, which are methods for obtaining tourist products
(such as events and Point of Interests (POI)) and tour
recommendation algorithms that create the 
itineraries~\cite{Lim2018}. The following sections
discuss related work in each field, respectively.

\subsubsection{Methods of retrieving travel products}

Before producing an itinerary, RSs have to formulate a
dataset of POIs from a data source. The proposed tour
recommendation algorithm will then evaluate a timetable
from this dataset after understanding the users' implicit
preferences. Throughout all the studies, there have been
several ways to identify an appropriate data source
representing real-life tourist trajectories.


One approach is made by gathering tourist products by
mining them from geotagged images such as Flickr, Facebook
or Twitter~\cite{DeChoudhury2010, Memon2015, Lucchese2012,
Lim2018a, HuiLim, HuiLima, Kurashima2013, Kurashima2010,
Brilhante2013, Brilhante2015 }. Lim et al.~\cite{Lim2018} 
denote this process into three steps; 


\begin{enumerate}

\item First, the application assembles an organised
    series of relevant photographs of the user’s
    destination.

\item The application then maps these pictures with a
    list of popular places extracted from sites like
    Wikipedia.

\item Since the photos contain each location’s
    timestamps; the application can calculate an
    approximate visit duration for each specific POI.\

\end{enumerate}

A quick and accurate approach towards gathering
essential places in the vicinity is using Mapping \&
Location APIs such as Foursquare's, Google's or TripAdvisor's. 
Wörndl et al.~\cite{Worndl2017} use this approach and build
a dataset of prominent POIs by querying the API with the
user's desired location. In return, they receive a
sequence of places and information about each site,
including its category, other user's ratings, opening
hours and helpful additional information that they can
use as criteria for the itineraries.

The ubiquitous presence of smartphones and GPS-enabled
devices has facilitated another approach to collecting
trajectories in a modern person's
life.~\cite{Chen2011, 10.1145/1889681.1889683} A
system can automatically gather the best POIs to visit based on
other users' historical paths providing more
information than previous methods such as the average
time people spend at a specific POI and how many
people go there. However, privacy issues are the main
caveat towards this approach since it requires people
to share their location constantly and publically\cite{Lim2018}. 

The most straightforward approach is by self-defining
the POIs or gathering them from a dataset such as the
TSPLIB95~\footnote{Sample instances for travelling
salesman Problem:
http://comopt.ifi.uni-heidelberg.de/software/TSPLIB95/}.
Building a
dataset from manually gathering POIs allows for
precision and a complete understanding of the
itinerary that the algorithm will generate. However,
the algorithm would be location-specific testified,
and the database would be personalised towards the
author's preferences~\cite{Chou2021a, Wisittipanich2020, Erbil}.

\subsubsection{Tour recommendation systems for both
individual and grouped travellers}

hhello

\subsection{User Profiling for Travel Preferences}
hello
